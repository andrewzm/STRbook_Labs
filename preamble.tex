%% MARGINS FROM ROB
% 6 1/8 x 9 1/4: 4.75 x 7.5
% 7 x 10: 5.6 x 8.6
% 7 1/2 x 9 1/4: 6 x 7.5

\newif\ifcomments
\newif\ifchecktext

%\commentsfalse
\commentstrue
%\checktextfalse
\checktexttrue

\usepackage{standalone}
\usepackage{newclude}
%\topmargin=-.5in
\topmargin=-.550in
\oddsidemargin=.25in
\evensidemargin=.25in
%\usepackage[emindex]{apacite}
\usepackage{amssymb,amsmath,amstext}
\usepackage{algorithm}
\usepackage{algpseudocode}
\usepackage{color}
\usepackage{dcolumn}
\usepackage{dsfont}
\usepackage{emptypage}
\usepackage{epsfig}
\usepackage[T1]{fontenc}
\usepackage{ifthen}
\usepackage{longtable}
\usepackage{index}
\usepackage{natbib}
\usepackage{setspace}
\usepackage{multirow}
\usepackage{tabularx}
\usepackage{threeparttable}
\usepackage{times}
\usepackage{titlesec}
\usepackage[nottoc,notlot,notlof]{tocbibind}
%\usepackage{tocbibind}
\usepackage{url}
\usepackage{verbatim}

%\widowpenalty10000
%\clubpenalty10000


\let\origdoublepage\cleardoublepage
\newcommand{\clearemptydoublepage}{%
	\clearpage
	{\pagestyle{empty}\origdoublepage}%
}

\newif\ifstandalone
\newif\ifstandalonechapter

%\usepackage{doublespace}

%\newcommand\myTwoFig[2]{%
%  \raisebox{-\height}{#1}%
%  \raisebox{-\height}{#2}%
%}
%\def\rf{\par\noindent\hangindent\parindent\hangafter1}

\usepackage[most]{tcolorbox}% http://ctan.org/pkg/tcolorbox

\newtheorem{technote}{Technical Note}[chapter]

\newtheorem{algorithmnote}{Algorithm}[chapter]
\newtheorem{algorithmnoteapp}{Algorithm}[section]

% A special box for this
\newtcolorbox{highlightbox}[1][]{colback=yellow, align=center,boxrule=0pt,boxsep=0pt,#1}

\usepackage[margin=0.75in,includefoot,includehead, paperwidth = 178mm, paperheight = 254mm]{geometry}

\usepackage{fancyhdr}
\pagestyle{fancy}
\fancyhf{} % sets both header and footer to nothing
\renewcommand{\headrulewidth}{0pt}
\renewcommand{\sectionmark}[1]{}
%\fancyfoot[CE,CO]{\copyright~C.K. Wikle, A. Zammit-Mangion, N. Cressie}
\fancyhead[LE,RO]{\thepage}
\fancypagestyle{Rfunfooter}{
  \pagestyle{fancy}
  %\fancyhf{}
  \renewcommand{\footrulewidth}{0.5pt}
   \fancyfoot[L]{{\small{Pages refer to the end of the code chunk containing the function. The function may hence appear on the page preceding that listed when the code chunk spans more than one page.}}}
   \fancyhead[LE,RO]{\thepage}
   \fancyhead[RE,LO]{\leftmark}
}

\fancypagestyle{Rfunfootermain}{
  \pagestyle{fancy}
  %\fancyhf{}
  \renewcommand{\footrulewidth}{0.5pt}
   \fancyfoot[L]{{\small{Pages refer to the end of the code chunk containing the function. The function may hence appear on the page preceding that listed when the code chunk spans more than one page.}}}
   \fancyhead[LE,RO]{ }
   \fancyhead[RE,LO]{ }
}


\newcommand{\wrapcolorbox}[2][]{%
  \colorbox{#1}{\parbox{\dimexpr\linewidth-2\fboxsep}{#2}}
}


\definecolor{mycolor}{rgb}{0.122, 0.435, 0.698}% Rule colour
\definecolor{Rtipcolor}{rgb}{0.122, 0.435, 0.698}% Rule colour
\definecolor{Advcolor}{rgb}{0.422, 0.135, 0.698}% Rule colour
\makeatletter
\newcommand{\Rtip}[1]{%
  \vspace{0.2in}
  \begin{tcolorbox}[colframe=Rtipcolor,boxrule=0.5pt,arc=4pt,
      left=6pt,right=6pt,top=6pt,bottom=6pt,boxsep=0pt,width=\linewidth,breakable]
    #1
  \end{tcolorbox}
  \vspace{0.2in}
}
\newcommand{\Adv}[1]{%
  \vspace{0.2in}
  \begin{tcolorbox}[colframe=Advcolor,boxrule=0.5pt,arc=4pt,
      left=6pt,right=6pt,top=6pt,bottom=6pt,boxsep=0pt,width=\linewidth,breakable]
    #1
  \end{tcolorbox}
  \vspace{0.2in}
}


\ifcomments
  \newcommand{\red}{\textcolor{red}}
\else
  \newcommand{\red}[1]{\null}
\fi


\ifchecktext
  \newcommand{\tcheck}{\textcolor{blue}}
\else
  \newcommand{\tcheck}{\textcolor{black}}
\fi


\newcommand{\mbf}[1]{\mathbf{#1}}
\newcommand{\mbv}[1]{\mbox{\boldmath$#1$\unboldmath}}
\newcommand{\bfu}{{ \bf u}}
\def\ba{\mathbf{a}}
\def\bb{\mathbf{b}}
\def\bc{\mathbf{c}}
\def\bd{\mathbf{d}}
\def\be{\mathbf{e}}
\def\bg{\mathbf{g}}
\def\bh{\mathbf{h}}
\def\bk{\mathbf{k}}
\def\bm{\mathbf{m}}
\def\br{\mathbf{r}}
\def\bs{\mathbf{s}}
\def\bu{\mathbf{u}}
\def\bv{\mathbf{v}}
\def\bw{\mathbf{w}}
\def\bx{\mathbf{x}}
\def\by{\mathbf{y}}
\def\bz{\mathbf{z}}
\def\bA{\mathbf{A}}
\def\bB{\mathbf{B}}
\def\bC{\mathbf{C}}
\def\bD{\mathbf{D}}
\def\bE{\mathbf{E}}
\def\bF{\mathbf{F}}
\def\bG{\mathbf{G}}
\def\bH{\mathbf{H}}
\def\bI{\mathbf{I}}
\def\bJ{\mathbf{J}}
\def\bK{\mathbf{K}}
\def\bL{\mathbf{L}}
\def\bM{\mathbf{M}}
\def\bP{\mathbf{P}}
\def\bQ{\mathbf{Q}}
\def\bR{\mathbf{R}}
\def\bS{\mathbf{S}}
\def\bT{\mathbf{T}}
\def\bU{\mathbf{U}}
\def\bV{\mathbf{V}}
\def\bW{\mathbf{W}}
\def\bX{\mathbf{X}}
\def\bY{\mathbf{Y}}
\def\bZ{\mathbf{Z}}
\def\bfzero{\mathbf{0}}
\def\bfone{\mathbf{1}}

\newcommand{\intd}{\textrm{d}}
\newcommand{\bfomega}{\mbox{\boldmath $\omega$}}
\newcommand{\bfalpha}{\mbox{\boldmath $\alpha$}}
\newcommand{\bfbeta}{\mbox{\boldmath $\beta$}}
\newcommand{\bfxi}{\mbox{\boldmath $\xi$}}
\newcommand{\bfeta}{\mbox{\boldmath $\eta$}}
\newcommand{\bftau}{\mbox{\boldmath $\tau$}}
\newcommand{\bfdelta}{\mbox{\boldmath $\delta$}}
\newcommand{\bftheta}{\mbox{\boldmath $\theta$}}
\newcommand{\bfTheta}{\mbox{\boldmath $\Theta$}}
\newcommand{\bfell}{\mbox{\boldmath $\ell$}}
\newcommand{\bfepsilon}{\mbox{\boldmath $\varepsilon$}}
\newcommand{\bfPhi}{\mbox{\boldmath $\Phi$}}
\newcommand{\bfPsi}{\mbox{\boldmath $\Psi$}}
\newcommand{\bfpsi}{\mbox{\boldmath $\psi$}}
\newcommand{\bfphi}{\mbox{\boldmath $\phi$}}
\newcommand{\bfmu}{\mbox{\boldmath $\mu$}}
\newcommand{\bfsigma}{\mbox{\boldmath $\sigma$}}
\newcommand{\bfnu}{\mbox{\boldmath $\nu$}}
\newcommand{\bfgamma}{\mbox{\boldmath $\gamma$}}
\newcommand{\bflambda}{\mbox{\boldmath $\lambda$}}
\newcommand{\bfLambda}{\mbox{\boldmath $\Lambda$}}
\newcommand{\bfSigma}{\mbox{\boldmath $\Sigma$}}
\newcommand{\bfPi}{\mbox{\boldmath $\Pi$}}
\newcommand{\var}{\textrm{var}}
\newcommand{\cov}{\textrm{cov}}
\newcommand{\bdiag}{\textrm{bdiag}}
\newcommand{\corr}{\textrm{corr}}
\newcommand{\pr}{\textrm{Pr}}
\newcommand{\fn}[1]{\texttt{\hlkwd{#1}}}
\newcommand{\num}[1]{\texttt{\hlnum{#1}}}
\newcommand{\strn}[1]{\texttt{\hlstr{#1}}}
\newcommand{\args}[1]{\texttt{\hlkwc{#1}}}
\newcommand{\R}{\texttt{R}}
\newcommand{\cc}[1]{\texttt{#1}}
\def\deg{$^{\circ}$ }
\newcommand{\rf}{\vskip .1in\par\sloppy\hangindent=1pc\hangafter=1
                 \noindent}


\textwidth= 5.5in
\textheight = 7.7in

\newcommand{\indexprint}[1]{#1\index{#1}}
%\makeindex[name=person,title={Index of persons}]
\makeindex
\newindex{aut}{adx}{and}{AuthorIndex}
\newindex{Rfun}{adx2}{and2}{R Function Index}

\DeclareMathOperator{\diag}{diag}
\DeclareMathOperator{\Gau}{Gau}
\DeclareMathOperator{\trace}{trace}
\DeclareMathOperator{\tr}{tr}

\let\MakeUppercase\relax

%% \usepackage{xwatermark}
%% \usepackage{tikz}
%% \newsavebox\mybox
%% \savebox\mybox{\tikz[color=black,opacity=0.1]\node{$\mathcal{Blah}$};}
%% \newwatermark*[
%%   allpages,
%%   angle=0,
%%   scale=2.4,
%%   xpos=44,
%%   ypos=53
%% ]{\usebox\mybox}

%\usepackage{draftwatermark}
%\SetWatermarkText{\begin{tabular}{c}  Preprint \\ 28 November 2018 \\  \copyright~Wikle, Zammit-Mangion, Cressie \\ Do Not Distribute! \end{tabular}}
%\SetWatermarkScale{0.3}
%\SetWatermarkColor{gray!20}

\bibliographystyle{asa_edition}
